\documentclass[twocolumn]{article}
\usepackage{ctex}
\usepackage{indentfirst}
\usepackage{amsmath}
\usepackage{algorithm}
\usepackage{algorithmic}
\usepackage{authblk}
\usepackage{graphicx}
\usepackage{listings}
\usepackage[utf8]{inputenc}
%\usepackage[T1]{fontenc}
%\usepackage{textcomp}
\usepackage{xcolor}
\usepackage[a4paper,scale=0.8]{geometry}  
\usepackage{color}
\lstdefinestyle{style1}{
    language=c,
    breaklines      =   true,
    columns         =   fixed, 
    breakatwhitespace=true,                          
    captionpos=b,                     
    keepspaces=true,                 
    numbers=right,                    
    numbersep=5pt,                  
    showspaces=false,                
    showstringspaces=false,
    showtabs=false,                  
    tabsize=2,
    flexiblecolumns,
    frame=lrtb              
}


\setlength{\parindent}{2em}
\title{2022秋compiler--lab1}
\author{学号: 201220154\ \ \ 姓名: 王紫萁\thanks{E-mail: ziqiwang.wayne@gmail.com}}
\date{\today}
\affil{Department of Computer Science, Nanjing University}
\begin{document}
\maketitle
\section{实验目标}
\subsection{词法分析器}
作为NJU-compiler制作之旅的开始,我们首先需要将输入文件中的所有词法单元
提取出来,c--的所有词法见Appendix\_A\cite{b1}。词法单元的提取利用DFA的思想实现,
而正则文法描述的语言正是他所对应的DFA能够接受的语言。

\subsection{语法分析器}
得到词法单元后,下一步就需要实现语法分析,通俗地,每个词素应该如何摆放到合适的位置上。
为了实现这一点,最强大的工具就是巴科斯范式表达的语法产生式,以及他所生成的语法树(AST)

\section{实验内容}
\subsection{实验环境}
\begin{itemize}
    \item Ubuntu 18.04.6 LTS on WSL2
    \item GCC version 7.5.0
    \item GNU Flex version 2.6.4
    \item GNU Bison version 3.0.4
\end{itemize}
\subsection{构建词法分析器}
不同的词素有不同的词法,我们主要介绍int型,float型以及块级注释的正则表达式。
int型有十进制,十六进制和八进制的写法。十进制是0或不以0开头的数字串,八进制是以0开头、
长度不小于2(包括0)且数字字面值为0-7以内的数字串,十六进制是以0x开头长度不小于3(包括0x)的数字串。
写作正则表达式为
\begin{lstlisting}[style=style1]
 HEX 0x[0-9a-fA-F]+
 OCT 0[0-7]+
 DEC 0|([1-9][0-9]*)
\end{lstlisting}

小数有科学计数法和常规表示法两种。常规表示法需要注意小数点两侧至少出现一个数字,
且整数部分可以有任意多的0。科学计数表示法是有如下格式的数字串
\begin{lstlisting}[style=style1]
 (({digit}+\.{digit}*)|\.{digit}+) ([Ee][+-]?{digit}+)?
\end{lstlisting}
其中e|E之前的是基数部分,之后是指数部分,且指数部分包括e|E可以不显示,基数部分就是常规表示法的格式。

最困难的当属块级注释表达式的书写了。先给出正确的写法:
\begin{lstlisting}[style=style1]
 \/\*(?:[^\*]|\*+[^\/\*])*\*+\/
\end{lstlisting}
首先需要熟知块级注释不允许嵌套,即“/*/**/*/”的形式是不被
允许的。表达式需要由/*开头,一个以上的*以及一个/结束。
来到最复杂的注释体部分,?:是为支持该语法的解析器加速匹配的控制符。
主体部分由除*以外的字符或一个* + 一个过多个* + 除/*以外的字符三部分组成。
主体部分重复一次或多次就能得到完整的注释内容。

\subsection{构建语法分析器}
在词法分析后我们得到了每一个词素所在的词法类型,他们作为终结符用在语法分析中。
Appendix\_A中有关于每个非终结符的产生式,产生式的书写不做赘述。值得一提的是有关
unary minus和binary minus的优先级确定问题。在教程中介绍了如何指定产生式的优先级来避免二义性,这给了我们启发,
即让uminus的优先级高于bminus即可,另外还需要说明uminus是左结合的(与终结符一致)。

其次,在int和float的产生式中
并没有关于+-符号的匹配处理(即我们的int和float实际上是无符号的),expression的产生式中也仅有uminus的匹配项。
这给我们后期处理字符串转数值型减少了判断条件,但也让编译器缺少处理uplus的能力。解决方法和uminus的处理方法类似。

最后有必要说明一下错误恢复的过程。在flex中,error是一个特殊的标记用来指明一个出错的非终结符。
这样在规约时如果遇到错误就能用error所在的产生式进行匹配。从而保证语法分析的继续执行。
在实验过程中,最易出错的点是在函数体和if-else的错误恢复中。即以下两句产生式:
$$
    \begin{aligned}
         & ExtDef \to Specifier\ error\ CompSt \\
         & Stmt \to IF\ error\ ELSE\ Stmt
    \end{aligned}
$$

错误原因在于,原先位置上的非终结符$FunDec$以及$Exp$均处于AST的较低层。
因此他们首先检测到错误后将错误向上传递。
如果不添加上述两句产生式,缺失前者会在形如\verb|int main{}|的输入串fail掉从而无法读取函数体内的内容,而缺失后者
可能会产生多余的报错信息,原因在于上文的错误未被跳过,影响到了下文错误的推断。
总而言之,错误恢复需要自顶向下分析,将易出现问题的非终结符所在的产生式之后加入在相同位置用error替换的新产生式,
经过不断尝试最终在合适的位置报告出合适数量的语法错误。

\subsection{FLEX 与 BISON综合构建Parser}
我们在前面讲到,FLEX将提取到的词素归类后传送给Bison,而Bison需要用其作为终结符作为产生式的一部分进行匹配。
实验手册告诉我们可以借助yyval全局变量传递参数。利用好这一性质,就可以很方便地构建AST了。一个AST是一棵多叉树,
其节点声明如下:
\begin{lstlisting}[style=style1]
typedef struct tree_t {
    int no_line;        // line number of current node
    ValueType value_t;  // value type of the scanned node
    union {
        int v_int;
        float v_float;
        char *v_string;
    } value;
    struct tree_t *father;
    struct tree_t *first_child;
    struct tree_t *next_sibling;
    struct tree_t *prev_sibling;
} TNode;

\end{lstlisting}

采用子女-兄弟表示法建立多叉树,用联合类型表示节点的真实值。只有INT, FLOAT, TYPE, ID
类型的节点具有有效的真实值,其中TYPE的真实值是int|float,ID的真实值是ID的字面量,均存放在
v\_string字段中。

我们为AST建立了new\_node, add\_child, free\_node, show\_tree四个接口,分别表示
建立节点,为父节点添加子节点,释放节点,以及打印语法树。头文件中有详尽的用法注释,
在此不多介绍。

在flex源文件中,每识别到一个非空格非注释的词法单元后,就新建节点指明属性,同时如果有字面值
就将yytext加入节点(深拷贝),否则传入空指针。Bison源文件在每个非空串产生式的语义动作中
先创建左侧非终结符的节点,再将右侧所有节点加入到其子节点中。由于flex采用自底向上分析,因此右侧的所有节点
在加入时已经创建成功。

\subsection{编译与链接}
\verb|parser.y|文件中的定义部分包含了对\verb|lex.yy.c|文件的引用,\verb|parser.l|
包含了\verb|syntex.tab.h|的引用。前者是flex生成的c文件,后者是\verb|bison -d|选项拆分出的头文件,
包含\verb|tokens|的声明。因此最后只需要将\verb|main.c|, \verb|parser.y|, \verb|tree.c|链接编译即可。
即最终的编译指令如下:
\begin{lstlisting}[style=style1]
$(CC) ${CMAIN} ${YFC} ./src/tree.c -lfl -ly -o parser
\end{lstlisting}
其中CC CMAIN YFC分别是gcc, main.c, .y文件的预定义。

\section{总结}
本次实验实现了编译器中最基础的词法分析和语法分析部分,并为语义分析构建出语法树。
实现了识别科学计数、十六进制,八进制的附加功能,后续考虑加入对\verb|#include|的识别来实现多文件分析。

\begin{thebibliography}{00}
    \bibitem{b1}《编译原理实践与指导教程》\ 许畅等
    \bibitem{b2}http://gnu.ist.utl.pt/software/flex/flex.html
    \bibitem{b3}https://en.wikipedia.org/wiki/Backus-Naur\_form
\end{thebibliography}
\end{document}